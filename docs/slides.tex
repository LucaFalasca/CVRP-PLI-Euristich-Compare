\documentclass[compress]{beamer}
\usepackage[
    title={CVRP: MIP and Euristic comparison},
    subtitle={Confronto tra modelli MIP e euristiche per il CVRP},
    event={Progetto fine corso AMOD},
    author={L. Falasca},
    longauthor={Luca Falasca},
    email={luca.falasca@students.uniroma2.eu},
    institute={Tor Vergata},
    longinstitute={Università degli Studi di Roma Tor Vergata},
]{unislides}
\usepackage{graphicx} % Required for inserting images
\usepackage{minted}
\usepackage{algorithm}
\usepackage{hyperref}
\usepackage{adjustbox}
\usepackage{svg}
%\svgsetup{inkscapelatex=false}

\begin{document}

\begin{frame}[plain]
	\titlepage
\end{frame}
\section{Introduzione}

\subsection{Obiettivi}
\begin{frame}{\subsecname}
    L'obiettivo di questo progetto è confrontare le prestazioni di un modello MIP e di alcune euristiche per la risoluzione del problema del Vehicle Routing Problem con capacità (CVRP).

    \begin{itemize}
        \item Implementazione di un modello MIP per il CVRP
        \item Implementazione di alcune euristiche per il CVRP
        \item Confronto delle prestazioni in termini di qualità della soluzione e tempo di esecuzione
    \end{itemize}
\end{frame}

\subsection{Dati e risorse tecniche}
\begin{frame}{\subsecname}
    \begin{itemize}
        \item Dati: set di benchmark Augerat 1995 e XML100 2021
        \item Software: Python (amplpy), AMPL, Gurobi
        \item Hardware: AMD Ryzen 5 7530U (6 core / 12 threads), 16 GB RAM
    \end{itemize}
\end{frame}

\section{Metodologia e Algoritmi}

\begin{frame}{Metodologia}
    \begin{itemize}
        \item Limite di tempo di 5 minuti (300 secondi) per ogni esecuzione
        \item Lower bound per il modello MIP
        \item Valutazione della distanza dalla soluzione ottima (Optimality Gap)
    \end{itemize}
\end{frame}

\subsection{Modello MIP}
\begin{frame}{\subsecname}
    \begin{itemize}
        \item Modello MIP b-matching relax (Miller 1995)
        \item Iterazione del modello MIP con aggiunta dinamica di vincoli fino alla soluzione ottima o al limite di tempo.
        \begin{itemize}
            \item Vincoli di sottociclo (subtour elimination constraints) 
            \item Vincoli di capacità (capacity constraints)
        \end{itemize}
        \item Uso di euristiche con grafi per trovare i vincoli dalla soluzione intermedia.
    \end{itemize}
\end{frame}

\subsection{Euristiche}

\subsubsection{Clarke and Wright}

\begin{frame}{\subsubsecname}
\end{frame}

\subsubsection{Sweep}

\begin{frame}{\subsubsecname}
\end{frame}

\subsubsection{My euristich}
\begin{frame}{\subsubsecname}
    \begin{itemize}
        \item Variante del Nearest Neighbor
        \item Nodi vicini agli estremi
        \item Aggiunta iterativa di nodi al percorso
    \end{itemize}
\end{frame}

\section{Risultati}

\subsection{Augerat 1995}
\begin{frame}{\subsecname}
\end{frame}

\subsection{XML100 2021}
\begin{frame}{\subsecname}
\end{frame}

\end{document}
